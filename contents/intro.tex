% !TEX root = ../main.tex

\chapter{绪论}
本章从对律师行业破产领域的债权会议业务数字化的研究背景意义出发,系统阐述国内外研究最新研究成果,以及概述本文主要研究内容和工作内容和本文的主要结构。

\section{研究背景与意义}
本文提到的数字化主要是利用数字技术,对具体业务、场景的数字化改造。近年来,我国不断加快数字基础设施建设,推进工业互联网、人工智能、物联网、车联网、大数据、云计算、区块链等技术集成创新和融合应用,让数字化应用更加广泛深入到社会经济运行的各个层面。而律师行业作为一个传统行业,目前大部分还处于传统的手工业时代,信息靠纸质材料收集整理,存储靠库房堆积,十分影响效率。如今正处疫情期间,线下办公的方式受到了极大的限制,而业务不会因为限制的存在而减少,大量的业务堆积带来了极大的不便,律师行业的数字化变革迫在眉睫。

在破产领域中,债权人会议及相关表决的意义十分重大。债权人会议是由所有依法申报债权的债权人组成,以保障债权人共同利益为目的,为实现债权人的破产程序参与权,讨论决定有关破产事宜,表达债权人意志,协调债权人行为的破产议事机构。债权人会议具有极其重要的作用:

1. 债权人会议的存在有利于法院中立公正的定位。作为破产清算程序虽然具有不同于一般诉讼程序的特殊性,但人民法院的性质决定了其在该程序中仍应是处于中立裁判者的地位,而清算组作为破产财产的管理人,从一定意义上讲说,它是继受了破产企业的权利义务,具有一方当事人的象征,在另一方当事人缺失或虚位状态下,往往导致法院的越位或债权人利益的受损。

2. 有利于保护债权人利益,增强清算组责任,是完善破产监督机制的需要。清算工作的主体为清算组,但其结果与债权人利益休戚相关,故而其对清算组的监督才是有能动性的有效监督。

3. 有利于促进破产程序有序化,清算组应通过债权人会议征询并听取债权人多数意见,将其处于散沙无序状态的利益冲突变为具有内部制约机制的民主集中表决,往往能将债权人会议各成员之间的冲突解决在内部,有异议的,还可申请法院复议、裁决,使破产工作有序展开。

所有债权人均为债权人会议成员,债权人会议成员享有表决权,特殊情况除外,债权人会议的表决结果决定一个破产项目的最终走向,因此在破产程序中,召开债权人会议是不必可少的一环。而往往在破产案件中,债权人较为分散,很大可能来自于全国各地,如果只召开现场会议,显然有些单一和固化,需要全国各地的债权人汇聚到一地,召开成本过高不利于高效快捷的处理问题。并且在目前疫情的影响下,将全国各地的债权人召集到一起召开债权人会议显然不切实际,因此实现破产业务的数字化转型是必要的。

破产领域的债权人会议全面数字化的研究意义主要有以下三点:

1. 债权人会议全面数字化可以简化破产项目的流程。由于受新冠肺炎疫情影响,通过网络形式召开债权人会议成为管理人的主要选择。网络债权人会议所代表的破产案件线上化操作的优势是非常明显的——这种优势不仅体现在疫情期间,线上化操作无论程序设计还是设备使用,更加简便快捷和规范高效,代表破产业务未来的工作趋势。网络债权人会议是在网络时代下,利用网络技术对债权人会议召开方式的革新,有利于提高工作效率,有利于更好保障债权人权利。但网络债权人会议有别于传统债权人会议,产生了许多新的问题需要解决。债权会议系统有别于常规的会议系统,一个破产项目下的债权人数量级很容易达到千人级别,有时甚至更高,并且债权人会议和破产法律紧密挂钩,从数量级和程序两方面考虑常规的会议系统并不适用,因此需要设计开发针对性的债权会议系统。

2. 债权人会议全面数字化便于存储和利用破产项目产生的数据。数字化以数据为主要生产要素,将企业中所有的业务、生产、营销、客户等有价值的人、事、物全部转变为数字存储的数据,形成可存储、可计算、可分析的数据、信息、知识,并和企业获取的外部数据一起,通过对这些数据的实时分析、计算、应用可以指导企业生产、运营等各项业务。而每一次的债权会议都能产生大量的数据,这些数据可以产生极大的经济价值,传统的债权人会议通过纸质材料对数据进行记录,而后存放至库房,没有对这些数据进行有效的利用,这是一笔极大的损失。

3. 债权人会议全面数字化过程中需要对高并发进行针对性的设计。近年来随着计算机技术和网络技术的迅速崛起,互联网日渐深刻的在改变着人们的生产生活方式。互联网的发展带来了需求的激增、技术上的压力,系统架构也因此不断的演进、升级、迭代。从单一应用,到垂直拆分,到分布式服务,到 SOA ,以及现在火热的微服务架构。虽然解决了大部分的问题,但是即时访问量比较巨大时,还是存在一些问题,尤其是高并发还是需要进行大量的设计和研究。对债权会议系统来说,一场债权会议的人数通常是千人到万人级别,有些甚至达到十万人级别,如果同时进行多场会议,且会议表决通常是集中在一段时间内的,这就带来了高并发问题,而债权会议的数据无比重要,如果丢失重要数据或者服务突然停摆,则意味着客户的永久失去及巨额的经济损失,因此债权会议数字化过程中,高并发针对性设计的重要性不言而喻。

\section{国内外研究现状}
近年来国内外学者和工程师不断探索传统行业数字化转型的实施方法,芦珊\cite{Li1999}提出剖析企业数字化转型的需求、目标、重点领域,深挖企业数字化转型的共性问题,通过分析企业数字化转型的顶层设计、业务、技术、运营数字化转型具体需求,提出适应企业转型建设的基础框架体系,用以指导企业数字化转型规划实施。

随着数字化建设的快速发展,越来越多现实生活中的业务都转到了线上进行,传统的 Web 架构难以满足日趋复杂的软件要求和服务质量的要求。随着大量服务转入线上,许多业务的办理变得简洁而快速,相应的由于越来越多的客户转倒线上,各种服务的并发量都在逐渐提升,高并发已经成为了一种常见的问题。而传统的大型分布式系统采用的高并发解决方案不适用部署在单服务器上的中小型Web应用,对于特定服务,需要针对性的设计高并发架构和其他解决方案,例如黄素萍\cite{Huang}等人提出针对单服务器的网络应用提出了一种适用的流量控制方案,即结合缓存、消息队列、不可重入锁的网络流量高并发优化处理算法方案。该方案使Web应用系统较传统数据库操作数据方法的处理能力和效率有明显提升,支持并发较高时的同时访问与同步数据处理,提升系统的并发性和吞吐量,同时保证系统的可用性。陈明\cite{Chen}等人针对传统Web容器管理会话的方案已经难以满足系统高可用的诉求的问题,提出了一种SESSION共享的方法解决该问题。

目前暂时没有专门针对视频表决会议系统的研究。目前常规的视频会议系统有飞书会议系统,钉钉会议系统和腾讯会议系统。首先从数量级上讲,飞书会议系统最大支持 1000 人同时参会,钉钉会议系统最大支持 302 人同时参会,而腾讯会议企业版最高支持 2000 人同时参会,对于债权人会议人数动辄千人,偶尔上万的级别来说,常规会议系统在数量级上不能满足需求。另外,债权人会议的重中之重是进行表决,表决的类型和计算方法十分繁杂,和破产相关法律紧密挂钩,传统的会议系统在专业性上也不能满足需求,因此对于常规会议系统的研究并不适用于本文所针对的视频表决会议系统。虽然没有相关的文献,但目前国内已经涌现了一批服务于视频表决会议的债权会议平台,其中较成熟的债权会议平台有:钉钉平台网络债权人会议、京东债权人会议系统、上海市第三中级人民法院会议系统、华宇九品“e破通”系统等,但各个会议系统仍各自存在一些问题。钉钉平台网络债权人会议是通过加入钉钉群直播的方式进行,钉钉群的人数上限为 500 人,如果超过需要使用多群联播的方式,最多可以转发至 45 个群,即上限为 22500 人,并且群组过多会导致管理十分不便。京东债权人会议系统需要通过京东自研的 ME(行业版)软件参加会议,并没有专门针对表决的会议设置。上海市第三中级人民法院会议系统较为完备,但是三中院的会议系统中,表决结果的计算不准确,最终表决的通过与否需要管理人下载表决数据自行计算。而华宇九品“e破通”系统也存在着表决数据表格复杂,管理人填写十分困难的问题。总的来说,目前仍没有一个特别完善的针对债权会议的视频表决会议系统,因此对此课题进行研究具有一定的参考价值。

\section{主要研究内容和工作内容}
上文中提到在近年来,在疫情影响、新一代数字技术的快速发展以及国家对于数字化基础设施建设大力支持的影响下,债权人会议的全面数字化具有极其重要的意义,并且目前并没有较为完善的研究。

针对上述问题,本文的主要工作是研究债权人会议业务的全面数字化,具体来说,就是研究高并发视频表决会议系统的设计与实现。此前,我们已经进行了和债权人会议系统相关的开发。在另一篇论文中,王志远\cite{Wang2021}讲述了基于微服务架构的破产业务的运营管理系统的设计与实现,其中也涉及到债权人会议的研究。但是在他的实现中,债权人会议仅仅是作为破产业务运营管理系统的一个模块,并且存在业务数据和运营管理系统数据紧耦合、对于高并发的支持不到位、时效性不佳等问题,无法满足真正场景下的债权人会议业务。因此本课题将在展露原系统问题的情况下研究如何设计系统满足破产重整法律案件债权会议的需求以及具体如何实现。主要研究的内容有高并发场景下的债权会议系统架构设计,高并发场景下的债权会议系统的数据存储设计,高并发场景下数据库的高可用问题,以及债权会议系统前端设计和优化问题。主要工作内容是债权会议系统的设计以及债权会议系统的前后端开发及性能测试。

\section{本文结构}
本篇论文分为绪论、系统需求分析及架构设计、系统后端设计与实现、系统前端设计与优化、实验设计、总结与展望六部分:

1. 绪论。本节高并发视频表决会议系统的设计与实现的背景出发,阐述本课题研究的意义。同时,本节将对数字化转型,高并发系统的研究现状以及表决会议系统的研究现状这三方面进行概述,探索此领域的未来发展方向。本节还介绍了本课题的主要研究内容和工作内容,并简单介绍了本文的主要结构。

2. 系统需求分析及架构设计。本节将从视频表决会议的业务现状出发,分析用户需求,在需求分析的基础上,根据系统的需求进行针对性的架构设计。最终本节将确定系统所需要的功能性需求和非功能性需求和系统架构设计。

3. 系统后端设计与实现。本节将在需求分析的基础上,根据需求完成数据库设计,并且根据需求进行后端的设计以及介绍具体实现。

4. 系统前端设计与优化。本节将在需求分析的基础上,根据需求完成前端的设计并针对发现的问题进行相关优化。

5. 实验设计。本节将对系统进行的功能性测试,性能测试工作做相关描述。

6. 总结与展望。本节将总结系统设计与实现的工程,并提出系统优化方案和建议。

