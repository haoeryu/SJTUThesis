% !TEX root = ../main.tex

\chapter{绪论}
本章将概述本文主要研究内容和工作内容以及本文的主要结构。

\section{主要研究内容和工作内容}
近年来,在疫情影响、新一代数字技术的快速发展以及国家对于数字化基础设施建设大力支持的影响下,各行各业都加快进行数字化变革。在法律行业,数字化变革进展缓慢,大多数业务仍然局限于线下办理。在法律行业的破产领域中,债权会议及表决的结果影响着整个破产项目流程的走向,是破产项目极为重要的一环,而由于疫情的影响,线下的债权会议召开受到了限制,破产项目无法顺利进行,这将带来极大的经济损失,因此本文对债权会议的全面数字化进行了研究。

本论文的主要工作是高并发视频表决会议系统的设计与实现。本课题将研究如何设计系统满足破产重整法律案件债权会议的需求以及具体如何实现。主要研究的内容有高并发场景下的债权会议系统架构设计,高并发场景下的债权会议系统的数据存储设计,高并发场景下数据库的高可用问题,以及债权会议系统前端设计和优化问题。主要工作内容是债权会议系统的设计以及债权会议系统的前后端开发及性能测试。

\section{本文结构}
本篇论文分为绪论、背景介绍、系统需求分析及架构设计、系统后端设计与实现、系统前端设计与优化、实验设计、总结与展望七部分:

1. 绪论。本节主要介绍了本课题的主要研究内容和工作内容,并简单介绍了本文的主
要结构。

2. 背景介绍。本节高并发视频表决会议系统的设计与实现的背景出发,阐述本课题研究的意义。同时,本节将对高并发系统的研究现状以及国内表决会议系统的现状这两方面进行概述,探索此领域的未来发展方向。

3. 系统需求分析及架构设计。本节将从视频表决会议的业务现状出发,分析用户需求,在需求分析的基础上,根据系统的需求进行针对性的架构设计。最终本节将确定系统所需要的功能性需求和非功能性需求和系统架构设计。

4. 系统后端设计与实现。本节将在需求分析的基础上,根据需求完成数据库设计,并且根据需求进行后端的设计以及介绍具体实现。

5. 系统前端设计与优化。本节将在需求分析的基础上,根据需求完成前端的设计并针对发现的问题进行相关优化。同时,本章展示了系统的用户页面。

6. 实验设计。本节将对系统进行的功能性测试,性能测试工作做相关描述。

7. 总结与展望。本节将总结系统设计与实现的工程,并提出系统优化方案和建议。

