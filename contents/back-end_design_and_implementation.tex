% !TeX root = ../main.tex

\chapter{系统实时后端设计与实现}
本章节主要介绍系统实时后端针对不同业务做出的设计,包括功能设计优化和实现。
实时后端包含了所以具有实时性的服务,包括聊天室服务和实时表决服务。WebSocket后端使用Golang实现。

  \section{主要问题}
  实时后端包括实时表决服务和实时聊天服务。
  
  对于实时表决服务来讲,债权人表决后,债权人和管理人端的数据需要进行更新,实时写入和读取会带来数据一致性问题且每次都进行全量更新会带来性能损耗问题。

  而对于实时聊天服务来讲,在支持私聊的情况下,若聊天室同时服务数千人,则聊天室针对每个人都需要保存一份聊天数据,而每个人保存的聊天数据中有大量重叠的地方,这部分重叠的地方重复计算带来的性能损耗问题。

  \section{设计与实现}
  由于实时系统数据比较复杂,本节首先对系统的数据进行分类。针对实时写入和读取带来的数据一致性问题,通过数据存储的设计进行解决,在开会期间,会议的数据读取从 Redis集群中读取缓存数据,更新会先更新 Redis 集群中的数据,然后再进行持久化,在架构设计一节已经介绍了相关过程。而针对每次进行全量更新带来的性能损耗问题,由于仅对管理人端查看实时表决情况需要数据的全量更新,且此部分更新时可以容许在最终结果无误的情况下在短时间内有一定误差,因此针对性的设计了两种数据推送方式。对于实时聊天服务聊天数据重叠过多导致性能开销较大的问题,通过聊天室共享一个持续增长的消息数组,消息按序排列,控制每条对于每个用户的可见权限解决了每个用户保存一个消息数据带来的重复计算性能开销问题。下面将进行详细的介绍。

  \subsection{数据分类}
\begin{figure}[!htp]
  \centering
  \includegraphics[width=12cm]{dataClassification.png}
  \caption[数据分类]
    {数据类型分类示意图}
 \label{fig:dataClassification}
\end{figure}

如图~\ref{fig:dataClassification},影响实例外在表现的所有副作用数据统称为有状态数据。
有状态数据分为两类,一类是会话数据,一类是业务数据。其中会话数据及业务数据中的Entity数据存于 Redis 中,其余数据均只存于实例内存中。

会话数据为连接的元信息,例如用户 Id ,聊天室 Id , jwt 验证信息等等。
在用户刚刚与实例建立连接时,实例将初始化好用户的会话数据并将其暂存于 Redis 中。
会话数据为每个用户对每个服务的独有数据,包含用户、连接、权限三者的元信息。
用户下线后,会话数据将会被清除。
同用户异地登录,所建立的会话会将原有会话顶替下线(实时服务一账号只可一人用)。

业务数据为影响用户前端展示的数据。例如:某债权人的表决情况,某会议的总签到人数等。
业务数据又分为 View 和 Entity 两种数据。

Entity 为 Mongo 中存储的真实数据,实例会将其暂存于 Redis 中。View 数据是抽象概念,指将对某类用户前端视图展现的全量信息。例如:管理人可见的表决统计总表数据,或者整个聊天室的所有聊天内容和封禁状态。View 并非 Mongo 内的实体数据,往往是多个 Entity 的聚合数据,或者是全体 Entity 的子集。View 数据内部包含同源数据与聚合数据两类。

同源数据是全体 Entity 的子集,内存中为至 Entity 的索引。以保证完全与 Entity 数据保持一致。比如某债权人对某表决是否投票,投了同意还是反对。

聚合数据是由 Entity 聚合而出的数据,会随着某些 Entity 的改变而改变。比如某个会议下某议题的总同意金额。

同源数据和聚合数据采用不同结构存储。它们的区分是调和 “性能” 与 “数据一致性” 矛盾的基石,下节数据存储将会介绍。

\subsection{数据存储}

\begin{figure}[!htp]
  \centering
  \includegraphics[width=12cm]{dataSave.png}
  \caption[Entity存储示意]
    {Entity存储示意图}
 \label{fig:dataSave}
\end{figure}

不同类的业务数据存储形式不同。Entity 数据最终会持久化至 Mongo。Entity 被分散到三处。实例内存是 Redis 的 Cache,而 Redis 事实上是 Mongo 的 Cache。

实例必须无状态,其状态必须可从 Redis 恢复。因此实例内存至 Redis 为 Write Through 策略。本文的第 5.2.2 节的实验已表明,3 个节点组成的单 Master 的 Mongo 集群仅支持 700+ 左右的 QPS,为性能瓶颈,因此 Redis 至 Mongo 为 Write Back 策略。如图~\ref{fig:dataSave}所示。

View 数据只保存在实例内存中。View 内部存储的同源数据事实上只是到某些 Entity 的映射(id、指针等)。View 内部存储的聚合数据是对应的聚合后的计算值。由于 View 内部存储的事实上只是 Entity 的映射,保证 View 的数据一致性的前提下提高了性能。

\subsection{数据推送}

\begin{figure}[!htp]
  \centering
  \includegraphics[height=3cm]{dataPush.png}
  \caption[推送方式]
    {推送方式示意图}
 \label{fig:dataPush}
\end{figure}

实时系统的核心即后台向前端的数据推送模式。业务数据同时通过 Base 和 Patch 两种方式推送。

Patch 推送则是指对 Entity 与 View 的更新事件。Patch 推送产生的原因可能来自于用户,如某用户发送了一条消息,也可能来自于集群,如结束表决的定时任务到时间了。Patch 推送产生后,先更新 Entity,再更新 View,最后再进行推送。View 数据可通过有序的 Patch 推送计算出来。

Base 推送是指将某个 View 数据推送给前端。Base 推送产生的原因可能来自于用户请求,如初始化内容,也可能来自于服务端的定时推送,如管理人查看表决详情的服务端推送。

\subsection{数据更新}

Entity 数据的更新完全来源于 Patch 推送。

View 数据的更新来源于 Patch 推送以及定时的 Entity 重新聚合计算。Patch 推送可能会更新前端 View 中的 聚合数据。在该策略下,必须进行定时重新聚合计算。定时的重新聚合计算目的是进行 聚合数据 的修正。常规地,在定时重新聚合计算完成后,一个 Base 推送应该被发起。

同源数据并不会被主动更新,因为它是到 Entity 的引用,Entity 被更新则意味着同源数据被更新。

\subsection{数据流}
本节中 “连接” 指 WebSocket 协议的连接,而 “会话” 指用户对服务的请求上下文。
\subsubsection{用户建立会话数据流}
\begin{figure}[!htp]
  \centering
  \includegraphics[width=12cm]{createDataflow.png}
  \caption[建立数据流]
    {用户建立会话数据流示意图}
 \label{fig:createDataflow}
\end{figure}
用户如果要与实例交互,必须先成功建立连接,然后建立会话。
会话是否建立及会话是否存在的判定均以是否建立会话数据,是否存在会话数据为准。
一个用户在同一时刻对一个实时服务只能有一个会话。用户建立会话数据具体流程为:

1. 客户端向 WebSocket 实例请求搭建连接,实例拿起该用户的会话并且建立锁,然后向 Redis 请求检查用户会话数据:

\quad{}\quad{}a. 如果已经存在会话数据,并且通过会话数据判定连接未关闭,则转至第 2 步解决异地登陆。

\quad{}\quad{}b. 如果已经存在会话数据,并且通过会话数据判定连接已关闭,则转至第 4 步覆盖会话;

\quad{}\quad{}c. 如果检查发现不存在会话数据,则转至第 4 步建立会话;

2. 实例发现异地登录请求,通过 Redis 中的连接,通知占用会话的用户异地登陆的异常;

3. 实例通过清除 Redis 中的对应会话数据关闭与当前占用会话用户的连接,并且转至第 4 步覆盖会话;

4. 实例生成针对该用户的会话数据,与该用户建立连接,并将会话数据存入 Redis;

5. 实例保存会话数据至内存,同时保存当前连接。此时视为会话已经成功建立;

6. 实例放开该用户的会话建立锁,通知用户,会话建立成功,可以接受推送订阅和推送请求。
\subsubsection{用户请求数据流}
\begin{figure}[!htp]
  \centering
  \includegraphics[width=12cm]{requestDateflow.png}
  \caption[请求数据流]
    {用户请求数据流示意图}
 \label{fig:requestDateflow}
\end{figure}
框架数据流整体过程如下:

1. 客户端向 WebSocket 实例发送请求:

\quad{}\quad{}a. 如果 Redis 中已经缓存了相关数据,并且当前请求为读请求,则跳转至第 10 步;

\quad{}\quad{}b. 如果 Redis 中已经缓存了相关数据,并且当前请求为写请求,则跳转至第 2 步;

\quad{}\quad{}c. 如果 Redis 中没有相关数据的缓存数据,则直接跳转至第 4 步;

2. 根据写请求的具体要求,WebSocket 实例向 Redis 中写入/更新 Entity 和 View数据;

3. 根据写请求的具体要求,WebSocket 实例将相关数据写入实例内存中,转至第 10 步;

4. 若 Redis 没有缓存数据,则由 WebSocket 实例调用非实时服务后端相关服务接口;

5. 非实时服务后端向 Mongo 数据库请求获取相关 Entity 数据;

6. Mongo 数据将相关 Entity 数据返回至非实时服务后端中;

7. 非实时服务后端返回相关 Entity 数据,由 WebSocket 实例计算出相关 View 数据;

8. WebSocket 实例将相关 Entity 数据和 View 数据缓存到 Redis 中;

9. Redis 缓存 Entity 数据和 View 数据成功,WebSocket 也在实例内存中保存对应数据;

10. WebSocket 将请求结果返回给 Client。

  \subsection{WebSocket 业务通信协议}
  在系统的实际业务中,WebSocket 消息传输为双向的。任一端都是 client 和 server。
无论 client 或 server,WebSocket 的所有消息入口只有一个。消息又只发向数个业务中的一个。
并且,根据网络 e2e 通信原则,“at least once” 的请求在业务层必须作出显式响应。
因此,本节将给出所有 WebSocket 业务通信协议的规范和定义,如图~\ref{fig:wsxy}所示。

\begin{figure}[!htp]
    \centering
    \includegraphics[height=5cm,width=14cm]{wsxy.png}
    \caption{WebSocket 业务通信协议示意图}
    \label{fig:wsxy}
  \end{figure}

  \subsubsection{消息类型}
  Client 与 server 间通信只接受 JSON 格式的 “消息”。
  消息分为多种类型,一种是请求型,一种是响应型。

  请求型消息,可由 client 或 server 发起,向对方请求内容或推送内容,格式如下方代码展示。
  {\setmainfont{Courier New Bold}
\begin{lstlisting}
    // 请求型消息
    {
        "t": "req",    // type
        "i": string,   // 消息 id。为随机数。如果 id 为 0,则代表该请求无需对方响应
        "h": string,   // handler。即 server 对应处理方法名
        "d": object    // data
    }
 \end{lstlisting}}

 响应型消息,对请求型消息的回复,格式如下方代码展示。
 {\setmainfont{Courier New Bold}
 \begin{lstlisting}
    // 响应型消息
    {
        "t": "res",    // type
        "i": string,   // 所响应的请求型消息的 id
        "s": number,   // HTTP 状态码
        "d": object    // data
    }
  \end{lstlisting}}

  \subsubsection{消息处理}
  WebSocket 连接的每一进程都同为 server 和 client,且每一进程的所有消息同享一个入口。
对于每一个 WebSocket 进程:

1. 作为 server 时会注册一系列方法,用于处理 d 字段的数据。

2. 作为 client 时会发送一系列消息,并等待响应结果返回给上层。

两进程的通信如图~\ref{fig:wsConnect}所示,蓝色 server 和 client 为同一进程,而黄色为另一进程,它们互相通信。
\begin{figure}[!htp]
    \centering
    \includegraphics[height=7.5cm]{wsConnect.png}
    \caption{消息处理}
    \label{fig:wsConnect}
  \end{figure}

白色数字标识 server 服务流程:

1. 服务端收到信息,检测 t 字段为 req,即为请求;

2. 根据信息 h 字段,寻找对应的 handler 以处理 d 字段中的数据;

3. Handler 返回请求结果。注意,若未找到 handler,则用默认 handler,返回错误信息;

4. 使用本实例的 client,向黄色进程发送返回结果,流程结束;

绿色数字标识 client 需要 server 响应的请求流程:

1. client 发送请求消息。上下文进入等待状态。若超时后则 abort(非幂等请求除外);

2. 进程执行上文 server 相同流程。流程结束后,本进程收到对应请求的响应消息;

3. 将响应消息填充向处于等待状态中的请求上下文,结束等待。

对于不需要 server 响应的 client,其发送的请求中 i 字段为 0。

\subsubsection{Golang 端}
Golang 端中的一份 WebSocket 连接分为两个协程处理,分别进行读操作和写操作的同步。
一个连接所涉及的所有协程均采用 context 进行结束。

Golang 端内部有一个 pendings map,专门用于等待用户的响应型请求。
map 的 key 是请求 id,值是响应型消息的指针的 chan(信道)。

如果要写入请求型消息,对于要用户响应的消息,随机生成一个请求 id,向 pendings map 中注册一个 chan(信道),在该 chan(信道) 上进行等待。
对于不需要用户响应的请求型消息,保持请求 id 为 0,跳过 pendings map 流程,直接写入即可返回。

如果要写入响应型消息,直接写入即可。

对于收到的请求型消息,根据请求型消息的 h 内容,调用程序内不同的 handler。handler 处理后,生成对应的响应型消息,发送给写协程返还给用户。

对于收到的响应型消息,根据其消息 id,找到 pendings map 中正在等待的 chan(信道),向其中写入该消息,让等待消息的协程结束等待。

\subsubsection{JavaScript 端}
JavaScript 端由于浏览器为单线程执行,因此无需关心同步异步的问题。

JavaScript 端内部有两个 map,一个存所有的 handler,一个存所有在等待的 Promise,kv 结构与 Golang 端类似。

如果要写入请求型消息,如果需要返回,则创建一个随机数作为请求 id,再创建一个 Promise 放入等待 map 中,并返回这个 Promise。
如果不需要用户响应的请求型消息,则保持请求 id 为 0,跳过 Promise 放入等待 map 等过程,直接返回即可。

如果要写入响应型消息,直接写入连接即可。

对于收到的请求型消息,根据请求型消息的 h 内容,调用 handler map 中不同的 handler,待其处理后生成对应的响应型消息,直接写入连接中。

对于收到的响应型消息,在等待 map 中找到对应的 Promise,根据消息的 s 进行 resolve 或 reject。

JavaScript 端代码完全用 class 实现,与 React 交互时会产生函数作为状态无法正确更新的问题,后续需换用自定义 hooks 实现。

\subsection{实时聊天室服务}
聊天室服务用于支持表决会议在线聊天,考虑到聊天室支持私聊,因此,每个用户在聊天室中的可见内容不同,即它们的 View 互不相同。
若聊天室同时服务数千人,则服务内部有数千份 View。
如果对每个人进行完整的 Base 推送,根据性能测试结果不能达到预期。

开销来源于重复 marshal,每个用户的 View 中有大量相同数据。
实时聊天室不同于表决服务,每位用户在聊天室中所见的消息是增量的,每份 View 的内容都是增量的,并且每个人的 View 又很可能重叠。这会导致大量计算开销。

\subsubsection{优化设计}
\begin{figure}[!htp]
  \centering
  \includegraphics[height=7.5cm]{wsChatRoom.png}
  \caption{聊天室优化方案}
  \label{fig:wsChatRoom}
\end{figure}
考虑到聊天室数据的本质是一个只读的,持续增长的消息序列并且聊天室并不存在聚合数据,只有同源数据,不需要定时的重新聚合计算任务。
因此,对连接上同一个聊天室的每个用户而言,可以认为所有人共享一个 View,其内部包括所有人发送的所有信息,只不过在 Base 请求时每个人可见的内容不同。

根据这种考虑,聊天室无需存储每个用户的数千个 View,只需存储一个 View,即一个持续增长的数组。
在这个设想下,系统内每条 Message 都有了先后顺序,它们在数组中的索引有序且唯一。
Message 的先后顺序能让聊天室快速构造 Base 推送。
只有用户知晓其接受到了多少 Message,只需定时告知聊天室,即可反推出所缺失的 Message。
用户请求包括两个字段,自己的最新 Message 索引号,以及自己的已知缺失 Message 数组。
聊天室的响应即 Base,包括用户的已知缺失的,以及用户最新至聊天室最新之间的 Message,如图~\ref{fig:wsChatRoom}所示。


  \subsubsection{交互设计}
  考虑到所有用户共享一个 View,其包括所有用户的聊天发言和封禁状态,因此每个用户在前端也维护类似这样的数据结构。具体地,前端维护:

  1. messages 数组,用于存放所有对该用户可见的消息,消息按序排列,该消息为 null 意为该消息对该用户不可见,该消息为 undefined 意为该消息缺失。

  2. bannedUsers 数组,用于存放被禁止的用户的 Id,对于管理人来说此数组包括所有被禁止的债权人用户 Id,对债权人来说此数组包含自身 Id 或 为空。

对于一个用户来说,其不清楚不可见的信息是由于网络延迟未收到还是由于自己不可见。

对于自己不可见的 message,其 messages 数组对应 index 处为 null,即 JavaScript 语义中的 “有数据但为空”。

对于不确定是不是自己不可见,且又未收到的数据,其 messages 数组对应 index 处为 undefined,即 JavaScript 语义中的 ”不知道“。
对于每个 undefined 的数据,前端会在一定时间内发送 generateBase 请求向后端确认,到底是不是自己不可见,如果不是则返回相应数据。

对于债权人,这样的定时任务为 120s,因为单个债权人看不到其他债权人的发言,因此聊天室内消息不多,服务端向该用户的网络拥塞可能性低。

对于管理人,这样的定时任务为 10s,因为管理人几乎可以看见所有聊天内容,因此聊天室内的消息很多,必须得确保内部的消息的正确性和及时性。

\subsection{实时表决服务}

\subsubsection{实时表决服务流程设计}
以管理人为例,实时表决服务流程如图~\ref{fig:wsbjlc}所示,具体流程如下:

\begin{figure}[!htp]
    \centering
    \includegraphics[height=6cm]{wsbjlc.png}
    \caption{WebSocket 表决服务流程示意图}
    \label{fig:wsbjlc}
  \end{figure}

  1. 管理人进入会议界面,通过前端界面交互向非实时服务后端发送请求;

 2. 后端返回统计数据给前端,前端收到请求后将数据在前端渲染,生成视图,前端接收到数据后需要将数据持久化到客户端上,从而减少后端每次计算全量数据的压力。为了明确前端数据是否需要被更新,后端数据应当给统计数据维护一个类似版本号的字段,若发现版本号变更,则更新数据,重新从非实时服务后端拉取一遍全量数据;

3. 拉取全量数据后,前端向后端发送请求询问会议是否已结束;

4. 非实时服务后端返回会议结束标识,根据会议结束标识出现如下两种情况:

  \quad{}\quad{}a. 结束标识会议已结束,该会议所有数据已在第 2 步全部获取,该流程结束;
  
  \quad{}\quad{}b. 结束标识会议未结束,第 2 步获取的数据仅作为基础数据,进入第 5 步;
  
5. 会议未结束,则后续数据由管理人和实时后端建立 WebSocket 链接,等待数据推送;
  
6. WebSocket 获取会议相关 Entity,数据不存在,向非实时服务后端发起数据请求,只有 WebSocket 本地没有缓存全量 Entity 时才会向非实时服务后端返回全量的数据,其他时候直接依据内存 / Redis 返回的缓存计算结果;
  
7. 非实时服务后端返回所有的 Entity,WebSocket 将其缓存在 Redis 中,并缓存在内存中;
  
8. WebSocket 根据实际情况实时推送数据到前端,前端依据 WebSocket 推的数据进行更新,WebSocket 不会返回给前端全量的数据,而是一个数据片段,包含各个议题的表决情况、参会人数情况和被更新的数据(比如某自然人就议题 A 提了同意票)。

  \subsubsection{实时表决服务内部数据流和控制流流转}
  \begin{figure}[!htp]
    \centering
    \includegraphics[height=8cm, width=12cm]{wszqrlc.png}
    \caption{债权人投票流程示意图}
    \label{fig:wszqrlc}
  \end{figure}

  WebSocket 会在下列场合和非实时服务后端、前端(债权人端和管理人端)以及 Redis 服务进行通信:

  1. 当 WebSocket 的缓存 Redis 没有缓存会议任何数据时,WebSocket 首先要和 非实时服务后端通信,获取该会议下全量的 Entity,然后 WebSocket 将这些 Entity 全部存储到 Redis 中,同时在内存中存储一份,它们是一次会议下的 meta 数据;同时,WebSocket 服务会根据得到的 Entity 数据形成一份 View 数据,View 数据表示的是各个议程的统计信息,View 数据不需要持久化到 Redis 中;
  
  2. 对于一个债权人而言,假设此时一个债权人投了票,则投票的行为通过前端的 WebSocket 客户端发送到 WebSocket 服务,WebSocket 服务根据收到的投票行为首先更新 Redis 的内存,然后更新对应的内存 Entity 实例,然后将这次表决结果存储在一个 Dirty 中(Dirty 是一段时间内表决 Data 的集合),并更新 View 数据,流程如图~\ref{fig:wszqrlc}所示。

  3. 对于一个管理人而言,其在一开始就已经从非实时服务后端拿到了所有 Entity,因此只需要接收变化的数据和统计信息:

  \quad{}\quad{}a. 在 WebSocket 链接第一次创建成功时,WebSocket 会向管理人推一次 Base 数据,Base 数据依托于 Entity 数据生成,只记录了投了票/参了会的人员;

  \quad{}\quad{}b. 链接创建成功后,WebSocket 会每隔一段时间(5秒)会检查 Dirty 的数据,如果 Dirty 数据不为空,则向管理人发送 View 和 Dirty 的数据,若中间有数据发送失败,则连接断开。

  \section{本章小结}
  本章介绍了本系统实时后端部分的具体设计。对于实时后端,首先介绍了实时后端的数据分类、存储、推送、更新和数据流设计,然后对 WebSocket 的业务通信协议设计做出了说明,最后对实时聊天服务和实时表决服务的详细设计做出了说明。