% !TEX root = ../main.tex

\chapter{全文总结}
本章将对支持大规模高并发、实时反映表决情况的债权人会议系统的设计与实现的工作进行总结,并根据课题论文的研究成果展望该领域未来发展方向。

本次课题研究以法律业务数字化为基点开展了研究,并且设计了针对高并发、实时反映表决情况的债权人会议系统,解决了债权会议业务数字化的问题,使债权会议全面线上化。在具体研究的过程中,所涉及到的开发技术有:面向对象程序设计、网页制作技术、消息队列、高速缓存等,最终设计并实现了高并发债权人会议系统。具体研究内容有:

1. 通过与甲方律师不断商讨的方式了解用户对于债权人会议系统的需求。对于实际需求,采用软件工程需求描述语言确定了系统的功能性需求。明确了软件开发的目标和开发的标准。

2. 针对高并发债权人会议系统的设计工作。为了适应本系统的需求,设计了针对性的架构解决系统性能问题。

3. 系统的开发及测试分析。概述了系统的前后端具体设计和部分实现。展示核心功能功能
的运行界面。完成系统设计之后,对系统进行功能性测试和部分性能测试。

本系统是对传统行业产业数字化的一次尝试。通过将信息技术应用于传统法律行业,系统
帮助实现破产领域债权会议业务的数字化,极大提高了债权会议业务的进行效率。当然
系统本身也存在一些不足之处,比如系统的 Mongo 写入性能为本系统的性能瓶颈,若后续系统达到性能瓶颈,需要对 Mongo 的写入性能做出优化。

