% !TeX encoding = UTF-8

% 载入 SJTUThesis 模版
\documentclass[type=master]{sjtuthesis}
% 选项
%   type=[doctor|master|bachelor|course],     % 可选(默认:doctor),论文类型
%   zihao=[-4|5],                             % 可选(研究生默认:-4,本科默认:5),正文字号大小
%   lang=[zh|en],                             % 可选(默认:zh),论文的主要语言
%   review,                                   % 可选(默认:关闭),盲审模式
%   [twoside|oneside]                         % 可选(默认:twoside),单双页模式

% 论文基本配置,加载宏包等全局配置
% !TEX root = ./main.tex

\sjtusetup{
  %
  %******************************
  % 注意:
  %   1. 配置里面不要出现空行
  %   2. 不需要的配置信息可以删除
  %******************************
  %
  % 信息录入
  %
  info = {%
    %
    % 标题
    %
    title           = {债权人会议系统设计与开发},
    title*          = {Creditor meeting system design and development},
    %
    % 标题页标题
    %   可使用“\\”命令手动控制换行
    %
    % display-title   = {上海交通大学学位论文\\ \LaTeX{} 模板示例文档},
    % display-title*  = {A Sample Document \\ for \LaTeX-based SJTU Thesis Template},
    %
    % 页眉标题
    %
    % running-title   = {示例文档},
    % running-title*  = {Sample Document},
    %
    % 关键词
    %
    keywords        = {债权人会议, 视频表决系统, 高并发, 实时系统},
    keywords*       = {Creditor's rights and meeting, Video voting system, High concurrency, Real-time},
    %
    % 姓名
    %
    author          = {王\quad{}浩\quad{}宇},
    author*         = {Mo Mo},
    %
    % 指导教师
    %
    supervisor      = {吴\quad{}刚},
    supervisor*     = {Prof. Mou Mou},
    %
    % 副指导教师
    %
    % assisupervisor  = {某某教授},
    % assisupervisor* = {Prof. Uom Uom},
    %
    % 学号
    %
    id              = {5140719024},
    %
    % 学位
    %   本科生不需要填写
    %
    degree          = {工学硕士},
    degree*         = {Master of Engineering},
    %
    % 专业
    %
    major           = {软件工程},
    major*          = {A Very Important Major},
    %
    % 所属院系
    %
    department      = {电子信息与电气工程学院},
    department*     = {Depart of XXX},
    %
    % 课程名称
    %   仅课程论文适用
    %
    course          = {某某课程},
    %
    % 答辩日期
    %   使用 ISO 格式 (yyyy-mm-dd);默认为当前时间
    %
    % date            = {2014-12-17},
    %
    % 资助基金
    %
    % fund  = {
    %           {国家 973 项目 (No. 2025CB000000)},
    %           {国家自然科学基金 (No. 81120250000)},
    %         },
    % fund* = {
    %           {National Basic Research Program of China (Grant No. 2025CB000000)},
    %           {National Natural Science Foundation of China (Grant No. 81120250000)},
    %         },
  },
  %
  % 风格设置
  %
  style = {%
    %
    % 本科论文页眉 logo 颜色 (red/blue/black)
    %
    % header-logo-color = black,
  },
  %
  % 名称设置
  %
  name = {
    % bib               = {References},
    % acknowledgements  = {谢\hspace{\ccwd}辞},
    % publications      = {攻读学位期间完成的论文},
  },
}

% 使用 BibLaTeX 处理参考文献
%   biblatex-gb7714-2015 常用选项
%     gbnamefmt=lowercase     姓名大小写由输入信息确定
%     gbpub=false             禁用出版信息缺失处理
\usepackage[backend=biber,style=gb7714-2015]{biblatex}
% 文献表字体
% \renewcommand{\bibfont}{\zihao{-5}}
% 文献表条目间的间距
\setlength{\bibitemsep}{0pt}
% 导入参考文献数据库
\addbibresource{bibdata/thesis.bib}

% 定义图片文件目录与扩展名
\graphicspath{{figures/}}
\DeclareGraphicsExtensions{.pdf,.eps,.png,.jpg,.jpeg}

% 确定浮动对象的位置,可以使用 [H],强制将浮动对象放到这里(可能效果很差)
% \usepackage{float}

% 固定宽度的表格
% \usepackage{tabularx}

% 使用三线表:toprule,midrule,bottomrule。
\usepackage{booktabs}

% 表格中支持跨行
\usepackage{multirow}

% 表格中数字按小数点对齐
\usepackage{dcolumn}
\newcolumntype{d}[1]{D{.}{.}{#1}}

% 使用长表格
\usepackage{longtable}

% 附带脚注的表格
\usepackage{threeparttable}

% 附带脚注的长表格
\usepackage{threeparttablex}

% 算法环境宏包
\usepackage[ruled,vlined,linesnumbered]{algorithm2e}
% \usepackage{algorithm, algorithmicx, algpseudocode}

% 代码环境宏包
\usepackage{listings}
\lstnewenvironment{codeblock}[1][]%
  {\lstset{style=lstStyleCode,#1}}{}

% 物理科学和技术中使用的数学符号,定义了 \qty 命令,与 siunitx 3.0 有冲突
% \usepackage{physics}

% 直立体数学符号
\newcommand{\dd}{\mathop{}\!\mathrm{d}}
\newcommand{\ee}{\mathrm{e}}
\newcommand{\ii}{\mathrm{i}}
\newcommand{\jj}{\mathrm{j}}

% 国际单位制宏包
\usepackage{siunitx}[=v2]

% 代码块宏包
\usepackage{listings}
% 代码高亮宏包
\usepackage{xcolor}
\usepackage{color}
\definecolor{dkgreen}{rgb}{0,0.6,0}
\definecolor{gray}{rgb}{0.5,0.5,0.5}
\definecolor{mauve}{rgb}{0.58,0,0.82}
\lstset{frame=tb,
     language=Java,
     aboveskip=3mm,
     belowskip=3mm,
     showstringspaces=false,
     columns=flexible,
     basicstyle = \ttfamily\small,
     numbers=none,
     numberstyle=\tiny\color{gray},
     keywordstyle=\color{blue},
     commentstyle=\color{dkgreen},
     stringstyle=\color{mauve},
     breaklines=true,
     breakatwhitespace=true,
     tabsize=3
}

% 定理环境宏包
\usepackage{ntheorem}
% \usepackage{amsthm}

% 绘图宏包
\usepackage{tikz}
\usetikzlibrary{shapes.geometric, arrows}

% 一些文档中用到的 logo
\usepackage{hologo}
\newcommand{\XeTeX}{\hologo{XeTeX}}
\newcommand{\BibLaTeX}{\textsc{Bib}\LaTeX}

% 借用 ltxdoc 里面的几个命令方便写文档
\DeclareRobustCommand\cs[1]{\texttt{\char`\\#1}}
\providecommand\pkg[1]{{\sffamily#1}}

% 自定义命令

% E-mail
\newcommand{\email}[1]{\href{mailto:#1}{\texttt{#1}}}

% hyperref 宏包在最后调用
\usepackage{hyperref}

% 自动引用题注更正为中文
\def\equationautorefname{式}
\def\footnoteautorefname{脚注}
\def\itemautorefname{项}
\def\figureautorefname{图}
\def\tableautorefname{表}
\def\partautorefname{篇}
\def\appendixautorefname{附录}
\def\chapterautorefname{章}
\def\sectionautorefname{节}
\def\subsectionautorefname{小节}
\def\subsubsectionautorefname{小节}
\def\paragraphautorefname{段落}
\def\subparagraphautorefname{子段落}
\def\FancyVerbLineautorefname{行}
\def\theoremautorefname{定理}


\begin{document}

%TC:ignore

% 标题页
\maketitle

% 原创性声明及使用授权书
\copyrightpage
% 插入外置原创性声明及使用授权书
% \copyrightpage[scans/sample-copyright-old.pdf]

% 前置部分
\frontmatter

% 摘要
% !TEX root = ../main.tex

\begin{abstract}
  近年来,在疫情、新一代数字技术的快速发展以及国家对于数字化基础设施建设大力支持的影响下,各行各业都加快进行数字化变革。在破产领域中,债权人会议是破产程序中必不可少的一环,而由于现场会议召开的成本过高且十分受限于疫情,而传统会议系统又无法满足债权人会议需求,因此需要对债权人会议业务的全面数字化。
  
  本文针对债权人会议人数多、请求在短时间内密集、实时性要求高的特点提出针对高并发场景、实时反映表决情况的债权人会议系统的设计与开发方案,结合相关法律知识,贴合债权人会议的现实业务,通过使用 Kafka 进行流量削峰解和使用 Redis 做高速缓存解决高并发情况下带来的问题,通过对实时后端的针对设计让债权人会议系统的实时性更强,使债权人会议的线上召开更为流畅。
  
  目前对债权人会议系统的研究十分匮乏,本文在法律业务数字化、债权人会议系统的设计开发、高并发架构针对设计、实时业务开发等方面提供了一定的参考价值。通过实验,本系统性能达到预期效果,Mongo 写入性能为本系统性能瓶颈,之后优化方向可以从此方面着手。
\end{abstract}

\begin{abstract*}
  In recent years, under the influence of the epidemic, the rapid development of a new generation of digital technology and the country's strong support for the construction of digital infrastructure, all industries are accelerating digital transformation. In the field of bankruptcy, creditors' meeting is an essential part of the bankruptcy process. However, due to the high cost of on-site meeting, which is very limited by the epidemic, and the traditional meeting system cannot meet the needs of creditors' meeting, it is necessary to fully digitize the creditors' meeting business.



  In view of the characteristics of large number of creditors' meetings, intensive requests in a short time and high real-time requirements, this paper proposes the design and development scheme of creditors' meeting system for high concurrent scenarios and real-time reflection of voting situation, and combines relevant legal knowledge to fit the realistic business of creditors' meetings. Through the use of Kafka flow peak cutting solution and the use of Redis cache to solve the problem of high concurrency, through the design of the real-time back-end so that the real-time creditor meeting system is stronger, so that the online creditor meeting is more fluent.



  At present, the research of creditor meeting system is very scarce. This paper provides some reference value in legal business digitization, creditor meeting system design and development, high concurrency architecture design, real-time business development and so on.Through the experiment, the system performance achieves the desired effect, and the Mongo write performance is the bottleneck of the system performance, and then the optimization direction can start from this aspect.
\end{abstract*}


% 目录
\tableofcontents
% 插图索引
\listoffigures*
% 表格索引
\listoftables*
% 算法索引
\listofalgorithms*

% 符号对照表
% !TEX root = ../main.tex

\begin{nomenclature*}
\label{chap:symb}

\begin{longtable}{rl}
  $\epsilon$  & 介电常数  \\
  $\mu$       & 磁导率    \\
\end{longtable}

\end{nomenclature*}


%TC:endignore

% 主体部分
\mainmatter

% 正文内容
% !TEX root = ../main.tex

\chapter{绪论}
DevOps 这一理念最早



这是 \sjtuthesis 的sdfgh示例文档,基本上覆盖了模板中所有格式的设置。建议大家在使用模
板之前,除了阅读《\sjtuthesis\ 使用文档》,这个示例文档也最好能看一看。

\section{研究背景与意义}

\subsection{三级标题}

\subsubsection{四级标题}

Lorem ipsum dolor sit amet, consectetur adipiscing elit, sed do eiusmod tempor
incididunt ut labore et dolore magna aliqua. Ut enim ad minim veniam, quis
nostrud exercitation ullamco laboris nisi ut aliquip ex ea commodo consequat.
Duis aute irure dolor in reprehenderit in voluptate velit esse cillum dolore eu
fugiat nulla pariatur. Excepteur sint occaecat cupidatat non proident, sunt in
culpa qui officia deserunt mollit anim id est laborum.

\section{脚注}

Lorem ipsum dolor sit amet, consectetur adipiscing elit, sed do eiusmod tempor
incididunt ut labore et dolore magna aliqua. \footnote{Ut enim ad minim veniam,
quis nostrud exercitation ullamco laboris nisi ut aliquip ex ea commodo
consequat. Duis aute irure dolor in reprehenderit in voluptate velit esse cillum
dolore eu fugiat nulla pariatur.}

\section{字体}


上海交通大学是我国历史最悠久的高等学府之一,是教育部直属、教育部与上海市共建的全
国重点大学,是国家“七五”、“八五”重点建设和“211 工程”、“985 工程”的首批建
设高校。经过 115 年的不懈努力,上海交通大学已经成为一所“综合性、研究型、国际化”
的国内一流、国际知名大学,并正在向世界一流大学稳步迈进。

{\songti 十九世纪末,甲午战败,民族危难。中国近代著名实业家、教育家盛宣怀和一批
  有识之士秉持“自强首在储才,储才必先兴学”的信念,于 1896 年在上海创办了交通大
  学的前身——南洋公学。建校伊始,学校即坚持“求实学,务实业”的宗旨,以培养“第
  一等人才”为教育目标,精勤进取,笃行不倦,在二十世纪二三十年代已成为国内著名的
  高等学府,被誉为“东方MIT”。抗战时期,广大师生历尽艰难,移转租界,内迁重庆,
  坚持办学,不少学生投笔从戎,浴血沙场。解放前夕,广大师生积极投身民主革命,学校
  被誉为“民主堡垒”。}

{\heiti 新中国成立初期,为配合国家经济建设的需要,学校调整出相当一部分优势专业、
  师资设备,支持国内兄弟院校的发展。五十年代中期,学校又响应国家建设大西北的号
  召,根据国务院决定,部分迁往西安,分为交通大学上海部分和西安部分。1959 年 3月
  两部分同时被列为全国重点大学,7 月经国务院批准分别独立建制,交通大学上海部分启
  用“上海交通大学”校名。历经西迁、两地办学、独立办学等变迁,为构建新中国的高等
  教育体系,促进社会主义建设做出了重要贡献。六七十年代,学校先后归属国防科工委和
  六机部领导,积极投身国防人才培养和国防科研,为“两弹一星”和国防现代化做出了
  巨大贡献。}

{\kaishu 改革开放以来,学校以“敢为天下先”的精神,大胆推进改革:率先组成教授代
  表团访问美国,率先实行校内管理体制改革,率先接受海外友人巨资捐赠等,有力地推动
  了学校的教学科研改革。1984 年,邓小平同志亲切接见了学校领导和师生代表,对学校
  的各项改革给予了充分肯定。在国家和上海市的大力支持下,学校以“上水平、创一流”
  为目标,以学科建设为龙头,先后恢复和兴建了理科、管理学科、生命学科、法学和人文
  学科等。1999 年,上海农学院并入;2005 年,与上海第二医科大学强强合并。至此,学
  校完成了综合性大学的学科布局。近年来,通过国家“985 工程”和“211 工程”的建
  设,学校高层次人才日渐汇聚,科研实力快速提升,实现了向研究型大学的转变。与此同
  时,学校通过与美国密西根大学等世界一流大学的合作办学,实施国际化战略取得重要突
  破。1985 年开始闵行校区建设,历经 20 多年,已基本建设成设施完善,环境优美的现
  代化大学校园,并已完成了办学重心向闵行校区的转移。学校现有徐汇、闵行、法华、七
  宝和重庆南路(卢湾)5 个校区,总占地面积 4840 亩。通过一系列的改革和建设,学校
  的各项办学指标大幅度上升,实现了跨越式发展,整体实力显著增强,为建设世界一流大
  学奠定了坚实的基础。}

{\ifcsname fangsong\endcsname\fangsong\else[无 \cs{fangsong} 字体。]\fi 交通大学
  始终把人才培养作为办学的根本任务。一百多年来,学校为国家和社会培养了 20余万各
  类优秀人才,包括一批杰出的政治家、科学家、社会活动家、实业家、工程技术专家和医家,如江泽民wowowowowowowoow陆定一、丁关根、汪道涵、钱学森、吴文俊、徐光宪、张光斗、黄炎
  培、邵力子、李叔同、蔡锷、邹韬奋、陈敏章、王振义、陈竺等。在中国科学院、中国工
  程院院士中,有 200 余位交大校友;在国家 23 位“两弹一星”功臣中,有 6 位交大校
  友;在 18 位国家最高科学技术奖获得者中,有 3 位来自交大。交大创造了中国近现代
  发展史上的诸多“第一”:中国最早的内燃机、最早的电机、最早的中文打字机等;新中国
  第一艘万吨轮、第一艘核潜艇、第一艘气垫船、第一艘水翼艇、自主设计的第一代战斗
  机、第一枚运载火箭、第一颗人造卫星、第一例心脏二尖瓣分离术、第一例成功移植同种
  原位肝手术、第一例成功抢救大面积烧伤病人手术等,都凝聚着交大师生和校友的心血智
  慧。改革开放以来,一批年轻的校友已在世界各地、各行各业崭露头角。}

{\ifcsname lishu\endcsname\lishu\else[无 \cs{lishu} 字体。]\fi 截至 2011 年 12
  月 31 日,学校共有 24 个学院 / 直属系(另有继续教育学院、技术学院和国际教育学
  院),19 个直属单位,12 家附属医院,全日制本科生 16802 人、研究生24495 人(其
  中博士研究生 5059 人);有专任教师 2979 名,其中教授 835 名;中国科学院院士 15
  名,中国工程院院士 20 名,中组部“千人计划”49 名,“长江学者”95 名,国家杰出
  青年基金获得者 80 名,国家重点基础研究发展计划(973 计划)首席科学家 24名,国
  家重大科学研究计划首席科学家 9名,国家基金委创新研究群体 6 个,教育部创新团队
  17 个。}

{\ifcsname youyuan\endcsname\youyuan\else[无 \cs{youyuan} 字体。]\fi 学校现有本
  科专业 68 个,涵盖经济学、法学、文学、理学、工学、农学、医学、管理学和艺术等九
  个学科门类;拥有国家级教学及人才培养基地 7 个,国家级校外实践教育基地 5个,国
  家级实验教学示范中心 5 个,上海市实验教学示范中心 4 个;有国家级教学团队 8个,
  上海市教学团队 15 个;有国家级教学名师 7 人,上海市教学名师 35 人;有国家级精
  品课程 46 门,上海市精品课程 117 门;有国家级双语示范课程 7 门;2001、2005 和
  2009 年,作为第一完成单位,共获得国家级教学成果 37 项、上海市教学成果 157
  项。}

% !TEX root = ../main.tex

\chapter{背景介绍}

数字化是信息技术发展的高级阶段,是数字经济的主要驱动力,随着新一代数字技术的快速发展,各行各业利用数字技术创造了越来越多的价值,这又推动了各行业加快进行数字化变革。本章从对律师行业破产领域的债权会议全面数字化的研究背景意义出发,系统阐述国内外研究最新研究成果。

\section{研究背景与意义}

\subsection{研究的背景}
本文提到的数字化主要是利用数字技术,对具体业务、场景的数字化改造。数字技术革命推动了人类的数字化变革。人类社会的经济形态随着技术的进步不断演变,农耕技术开启了农业经济时代,工业革命实现了农业经济向工业经济的演变,如今数字技术革命,推动了人类生产生活的数字化变革,孕育出一种新的经济形态——数字经济,数字化成为数字经济的核心驱动力。数字技术的不断完善让数字化的价值得到充分发挥。近年来,物联网、云计算、人工智能等各类数字技术不断更新,从实验向实践,逐渐工业化,形成了完整的数字化价值链,在各个领域实现应用,推动了各个行业的数字化,为各行业不断创造新的价值。数字基础设施快速发展推动数字化的应用。近年来,我国不断加快数字基础设施建设,推进工业互联网、人工智能、物联网、车联网、大数据、云计算、区块链等技术集成创新和融合应用,让数字化应用更加广泛深入到社会经济运行的各个层面,推动数字经济的发展。

律师行业作为一个传统行业,目前大部分还处于传统的手工业时代,信息靠纸质材料收集整理,存储靠库房堆积,十分影响效率。如今正处疫情期间,线下办公的方式受到了极大的限制,而业务不会因为限制的存在而减少,大量的业务堆积带来了极大的不便,律师行业的数字化变革迫在眉睫。

而在破产领域中,会议和表决的意义十分重大,决定着一个破产项目的走向,为了实现破产业务的数字化转型,专门设计针对破产项目表决的会议系统是必不可少的。企业破产项目债权人很容易高达千人级别甚至万人级别,如果同时进行多场会议,且在同一时间段进行表决的情况下,并发量将极大的提升,对服务器造成极大的压力,因此需要设计开发专门针对高并发情况下的会议表决系统。

\subsection{研究的意义}
破产领域的债权会议全面数字化的研究意义主要有以下三点。

一是由于受新冠肺炎疫情影响,通过网络形式召开债权人会议成为管理人的主要选择。网络债权人会议所代表的破产案件线上化操作的优势是非常明显的——这种优势不仅体现在疫情期间,线上化操作无论程序设计还是设备使用,更加简便快捷和规范高效,代表破产业务未来的工作趋势。网络债权人会议是在网络时代下,利用网络技术对债权人会议召开方式的革新,有利于提高工作效率,有利于更好保障债权人权利。但网络债权人会议有别于传统债权人会议,产生了许多新的问题需要解决。债权会议系统有别于常规的会议系统,和破产法律紧密挂钩,常规的会议系统并不适用,因此需要设计开发针对性的债权会议系统。

二是数字化以数据为主要生产要素,将企业中所有的业务、生产、营销、客户等有价值的人、事、物全部转变为数字存储的数据,形成可存储、可计算、可分析的数据、信息、知识,并和企业获取的外部数据一起,通过对这些数据的实时分析、计算、应用可以指导企业生产、运营等各项业务。而每一次的债权会议都能产生大量的数据,这些数据可以产生极大的经济价值,因此需要保证对这些数据的存储的可靠性。

三是近年来随着计算机技术和网络技术的迅速崛起,互联网日渐深刻的在改变着人们的生产生活方式。互联网的发展带来了需求的激增、技术上的压力,系统架构也因此不断的演进、升级、迭代。从单一应用,到垂直拆分,到分布式服务,到 SOA ,以及现在火热的微服务架构。虽然解决了大部分的问题,但是即时访问量比较巨大时,还是存在一些问题,尤其是高并发和高可用还是需要进行大量的设计和研究。对债权会议系统来说,一场债权会议的人数通常是千人到万人级别,有些甚至达到十万人级别,如果同时进行多场会议,且会议表决通常是集中在一段时间内的,这就带来了高并发问题,而债权会议的数据无比重要,如果丢失重要数据或者服务突然停摆,则意味着客户的永久失去及大量的经济损失,因此债权会议数字化过程中,高并发和高可用的针对性设计的重要性不言而喻。





\input{contents/floats}
% !TEX root = ../main.tex

\chapter{全文总结}
本章将对支持大规模高并发、实时反映表决情况的债权人会议系统的设计与实现的工作进行总结,并根据课题论文的研究成果展望该领域未来发展方向。

本次课题研究以法律业务数字化为基点开展了研究,并且设计了针对高并发、实时反映表决情况的债权人会议系统,解决了债权会议业务数字化的问题,使债权会议全面线上化。在具体研究的过程中,所涉及到的开发技术有:面向对象程序设计、网页制作技术、消息队列、高速缓存等,最终设计并实现了高并发债权人会议系统。具体研究内容有:

1. 通过与甲方律师不断商讨的方式了解用户对于债权人会议系统的需求。对于实际需求,采用软件工程需求描述语言确定了系统的功能性需求。明确了软件开发的目标和开发的标准。

2. 针对高并发债权人会议系统的设计工作。为了适应本系统的需求,设计了针对性的架构解决系统性能问题。

3. 系统的开发及测试分析。概述了系统的前后端具体设计和部分实现。展示核心功能功能
的运行界面。完成系统设计之后,对系统进行功能性测试和部分性能测试。

本系统是对传统行业产业数字化的一次尝试。通过将信息技术应用于传统法律行业,系统
帮助实现破产领域债权会议业务的数字化,极大提高了债权会议业务的进行效率。当然
系统本身也存在一些不足之处,比如系统的 Mongo 写入性能为本系统的性能瓶颈,若后续系统达到性能瓶颈,需要对 Mongo 的写入性能做出优化。



%TC:ignore

% 参考文献
\printbibliography[heading=bibintoc]

% 附录
\appendix

% 附录中图表不加入索引
\captionsetup{list=no}

% 附录内容
\input{contents/app_maxwell_equations}
\input{contents/app_flow_chart}

% 结尾部分
\backmatter

% 用于盲审的论文需隐去致谢、发表论文、科研成果、简历

% 致谢
% !TEX root = ../main.tex

\begin{acknowledgements}
  感谢上海交通大学软件学院高质量的教学水平,开设诸多高质量的专业课,让我有机会在大学四年的时光中系统地学习计算机基础和软件工程等方面地知识,并能从事相关开发实践。在软件学院诸多优秀的老师和同学的陪伴下,我大学四年受益良多。

  感谢吴刚老师和任锐老师在我编写本论文过程中的指导。老师们在系统开发方面丰富的实
  践经历和全面的系统意识让我受益良多,他们严谨认真的科研精神也在时时刻刻地给予我正面的影响。通过和老师们的交流,我进一步学习了软件工程开发思想和软件系统架构设计的相关知识。感谢老师们能够在我在进行本课题研究遇到问题和疑惑时及时向我伸出援手。

  感谢 SAIL 实验室的林许亚伦学长和郑宇宸学长在我开发系统过程给予我的帮助。学长
  们十分热心负责,在开发过程中提出了非常多有用的建议,并且协助我解决了很多难题,他们的工作态度令人敬佩。

  感谢软件工程专业的蒋钊、龙泓杙两位同学这四年以来的帮助和陪伴。两位同学发现问题提出问题的能力,解决问题严谨认真的态度对我影响颇大,并且在学业和项目方面对我帮助颇多,在此对他们由衷的感谢。
\end{acknowledgements}


% 发表论文及科研成果
% 盲审论文中,发表论文及科研成果等仅以第几作者注明即可,不要出现作者或他人姓名
\input{contents/achievements}

% 简历
\input{contents/resume}

% 学士学位论文要求在最后有一个大摘要,单独编页码
% !TEX root = ../main.tex

\begin{digest}
  Digitalization is the advanced stage of the development of information technology and the main driving force of digital economy. With the rapid development of the new generation of digital technology, all walks of life create more and more value by using digital technology, which promotes all walks of life to accelerate the digital transformation.



The digitalization mentioned in this paper mainly refers to the digital transformation of specific business and scene by using digital technology. The digital technology revolution has promoted the digital transformation of mankind. The economy of human society with the progress of technology evolved, farming technology opens the agricultural economy, industrial revolution to realize the evolution of the agricultural economy to industrial economy, now the digital technology revolution, to promote the digital revolution of human production and living, gives birth to a new economic form - digital economy, digital driving force to become the core of the digital economy. The continuous improvement of digital technology gives full play to the value of digitalization. In recent years, various digital technologies such as the Internet of Things, cloud computing and artificial intelligence have been constantly updated and gradually industrialized from experiment to practice, forming a complete digital value chain that has been applied in various fields, promoting the digitalization of various industries and creating new values for all industries. The rapid development of digital infrastructure drives the adoption of digitization. In recent years, China has been accelerating the construction of digital infrastructure, promoting the integrated innovation and application of industrial Internet, artificial intelligence, Internet of Things, Internet of vehicles, big data, cloud computing, block chain and other technologies, so as to make digital application more widely and deeply into all levels of social and economic operation and promote the development of digital economy.



As a traditional industry, the lawyer industry is still in the traditional handicraft era. Information is collected and sorted out by paper materials and stored in warehouses, which greatly affects efficiency. At present, in the midst of the epidemic, offline work is greatly restricted, and business will not be reduced because of the existence of restrictions. A large amount of business accumulation has brought great inconvenience, and the digital reform of the lawyer industry is imminent.



In the field of bankruptcy, the significance of meeting and voting is very important, which determines the trend of a bankruptcy project. In order to realize the digital transformation of bankruptcy business, it is essential to design a meeting system for bankruptcy project voting. Creditors of enterprise bankruptcy project are easy to reach the level of thousands or even ten thousand. If multiple meetings are held at the same time and voting is conducted in the same period, the concurrency will be greatly increased, causing great pressure on the server. Therefore, it is necessary to design and develop a meeting voting system specifically for high concurrency.



The research significance of comprehensive digitalization of creditor's rights meeting in the field of bankruptcy mainly includes the following three points:

One is that due to the COVID-19 pandemic, holding creditors' meetings online has become the main choice of managers. The advantages of online operation of bankruptcy cases represented by the network creditors' meeting are very obvious -- this advantage is not only reflected in the epidemic period, online operation is more convenient, fast, standardized and efficient in both program design and equipment use, and represents the future work trend of bankruptcy business. Network creditor's meeting is an innovation of the way of holding creditor's meeting by using network technology in the network era, which is conducive to improving work efficiency and better protecting creditor's rights. However, the network creditors' meeting is different from the traditional creditors' meeting, and many new problems need to be solved. Creditor's rights conference system is different from the conventional conference system, which is closely linked with the bankruptcy law. The conventional conference system is not applicable, so it is necessary to design and develop a targeted creditor's rights conference system.



Second, digitalization takes data as the main factor of production, converts all valuable people, things and things in the enterprise, such as business, production, marketing and customers, into digital stored data, forming data, information and knowledge that can be stored, calculated and analyzed. Together with external data obtained by the enterprise, Through the real-time analysis, calculation and application of these data can guide enterprises' production, operation and other businesses. And each creditor's rights meeting can produce a large amount of data, which can produce great economic value, so it is necessary to ensure the reliability of the storage of these data.



Third, with the rapid rise of computer technology and network technology in recent years, the Internet is changing people's way of production and life. The development of the Internet has brought about a surge in demand and pressure on technology, so the system architecture is constantly evolving, upgrading and iterating. From single application, to vertical split, to distributed services, to SOA, and now the hot microservices architecture. Although most of the problems have been solved, there are still some problems even when the traffic volume is relatively large, especially high concurrency and high availability still need a lot of design and research. For creditor's rights conference system, a creditor meeting is usually one thousand to ten thousand the number of levels, some even reach one hundred thousand level, if for many meetings at the same time, and meeting voting is usually concentrated in a period of time, that leads to the high concurrency issues, and the creditor meeting data very important, if the loss of important data or service suddenly lockout, Therefore, in the process of digitalization of creditor's rights meeting, the importance of targeted design with high concurrency and high availability is self-evident.



The main work of this thesis is the design and implementation of high concurrent video voting conference system. This topic will study how to design a system to meet the needs of legal cases of bankruptcy reorganization creditor's rights meeting and how to achieve it. The main research contents include the architecture design of the debt conference system in high concurrency scenario, the data storage design of the debt conference system in high concurrency scenario, the high availability of database in high concurrency scenario, and the front-end design and optimization of the debt conference system. The main work content is the creditor's rights conference system design and the creditor's rights conference system front and back end development and performance test.
\end{digest}


%TC:endignore

\end{document}
